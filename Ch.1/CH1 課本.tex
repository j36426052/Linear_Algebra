%% LyX 2.3.5.2 created this file.  For more info, see http://www.lyx.org/.
%% Do not edit unless you really know what you are doing.
\documentclass[12pt,english]{article}
\usepackage{fontspec}
	\setmainfont{Cochin}
\usepackage{setspace}
	\onehalfspace
\usepackage{amssymb, amsmath, amsthm}
\usepackage{mathrsfs}
\usepackage{tasks}
\renewcommand{\ttdefault}{cmtl}
\usepackage{tcolorbox}
\usepackage[framemethod=tikz]{mdframed} % Allows defining custom boxed/framed environments
\usepackage[a4paper]{geometry}
\geometry{verbose,tmargin=2.5cm,bmargin=2.5cm,lmargin=3cm,rmargin=3cm}
\usepackage{enumitem}
\everymath{\displaystyle}

\makeatletter
%%%%%%%%%%%%%%%%%%%%%%%%%%%%%% Textclass specific LaTeX commands.
\theoremstyle{plain}
\newtheorem{thm}{\protect\theoremname}[section]
\newtheorem{cor}[thm]{Corollary}
\newtheorem{lmma}[thm]{Lemma}
\newtheorem*{defn}{\underline{Definition}}
\newtheorem*{ex*}{Example}
\newtheorem*{sol*}{Solution}
\newtheorem*{thm*}{Theorem}
\newtheorem*{lmma*}{Lemma}
\newtheorem*{rmk*}{Remark}
\newtheorem*{pf*}{\underline{\textbf{Proof\ }}}
%%%%%%%%%%%%%%%%%%%%%%%%%%%%%% User specified LaTeX commands.
\renewcommand{\P}{\mathscr{P}}
\newcommand{\B}{\mathscr{B}}
\newcommand{\A}{\mathscr{A}}
\newcommand{\C}{\mathbb{C}}
\newcommand{\CC}{\mathscr{C}}
\newcommand{\R}{\mathbb{R}}
\newcommand{\Q}{\mathbb{Q}}
\newcommand{\Z}{\mathbb{Z}}
\newcommand{\N}{\mathbb{N}}
\newcommand{\X}{\mathcal{X}}
\newcommand{\T}{\mathscr{T}}
\newcommand{\arbuni}{\bigcup_{\alpha\in I}}
\newcommand{\finint}{\bigcap_{i=1}^n}
\newcommand{\Ua}{{\textsc{U}_\alpha}}
\newcommand{\Ui}{\textsc{U}_i}

\newcommand{\pair}[2]{\left( \,#1\,,\,#2\,\right) }
\makeatother

\usepackage{babel}
\providecommand{\definitionname}{Definition}
\providecommand{\theoremname}{Theorem}
%----------------------------------------------------------------------------------------
%	INFORMATION ENVIRONMENT
%----------------------------------------------------------------------------------------

% Usage:
% \begin{info}[optional title, defaults to "Info:"]
% 	contents
% 	\end{info}

\mdfdefinestyle{info}{%
	topline=false, bottomline=false,
	leftline=false, rightline=false,
	nobreak,
	singleextra={%
		\fill[black](P-|O)circle[radius=0.4em];
		\node at(P-|O){\color{white}\scriptsize\bf i};
		\draw[very thick](P-|O)++(0,-0.8em)--(O);%--(O-|P);
	}
}

% Define a custom environment for information
\newenvironment{info}[1][Info:]{ % Set the default title to "Info:"
	\medskip
	\begin{mdframed}[style=info]
		\noindent{\textbf{#1}}
}{
	\end{mdframed}
}
%----------------------------------------------------------------------------------------
%	WARNING TEXT ENVIRONMENT
%----------------------------------------------------------------------------------------

% Usage:
% \begin{warn}[optional title, defaults to "Warning:"]
%	Contents
% \end{warn}

\mdfdefinestyle{warning}{
	topline=false, bottomline=false,
	leftline=false, rightline=false,
	nobreak,
	singleextra={%
		\draw(P-|O)++(-0.5em,0)node(tmp1){};
		\draw(P-|O)++(0.5em,0)node(tmp2){};
		\fill[black,rotate around={45:(P-|O)}](tmp1)rectangle(tmp2);
		\node at(P-|O){\color{white}\scriptsize\bf !};
		\draw[very thick](P-|O)++(0,-1em)--(O);%--(O-|P);
	}
}

% Define a custom environment for warning text
\newenvironment{warn}[1][Warning:]{ % Set the default warning to "Warning:"
	\medskip
	\begin{mdframed}[style=warning]
		\noindent{\textbf{#1}}
}{
	\end{mdframed}
}
%----------------------------------------------------------------------------------------
%	NUMBERED QUESTIONS ENVIRONMENT
%----------------------------------------------------------------------------------------

% Usage:
% \begin{question}[optional title]
%	Question contents
% \end{question}

\mdfdefinestyle{Example}{
	innertopmargin=1.2\baselineskip,
	innerbottommargin=0.8\baselineskip,
	roundcorner=5pt,
	nobreak,
	singleextra={%
		\draw(P-|O)node[xshift=1em,anchor=west,fill=white,draw,rounded corners=5pt]{%
		Example \exampleTitle};
	}
}

%\newcounter{Question} % Stores the current question number that gets iterated with each new question

% Define a custom environment for numbered questions
\newenvironment{example}[1][\unskip]{
	\bigskip
	%\stepcounter{Question}
	\newcommand{\exampleTitle}{~#1}
	\begin{mdframed}[style=Example]
}{
	\end{mdframed}
	\medskip
}








% Begin from here




\begin{document}

\section*{Ch1. Vector Space}

\begin{defn}[ Vector Space ]
$\\$	A vector space (or linear space) $\mathrm{W}$ over a $Field$ $\mathrm{F}$ consists of a set on which two operations (called addition and scalar multiplication, respectively) are defined so that for each pair of elements $x, y$, in $\mathrm{W}$ there is a unique element $x+y$ in $\mathrm{W}$, and for each element a in F and each element $x$ in $\mathrm{W}$ there is a unique element $ax$ in $\mathrm{W}$, such that the following conditions hold.

\end{defn}
\subsection*{\color{red}There are eight rules for calculating the vector space !\\Please do not forget !\\}
\subsection*{\S 1-3 Subspace}
\begin{defn}[ Definition of Subspace ]
$\\$
	A subset $\mathrm{W}$ of a vector space $\mathrm{W}$ over a field $\mathrm{F}$ is called a subspace of $\mathrm{W}$ if $\mathrm{W}$ is a vector space over $\mathrm{F}$ with the operations of addition and scalar multiplication defined on $\mathrm{W}$.\\
\end{defn}



\begin{thm*}[1.3] 
$\\$ Let V be a vector space and W a subset of V. Then W is a subspace of V if and only if the following three conditions hold for the operations defined in V.
\\
(a)\ \ $0 \in \mathrm{W}.$ \\
(b)\ \ $x+y \in W$ whenever $x \in \mathrm{W}$ and $y \in \mathrm{W}.$ \\
(c)\ \ $cx \in \mathrm{W}$ whenever $c \in \mathrm{F}$ and $x \in \mathrm{W}.$ \\

\end{thm*}

% Exercise 8:a,c,f


\begin{thm*}[1.4]
$\\$ Any intersection of subspaces of a vector space $\mathrm{W}$ is a subspace of $\mathrm{V}$.
	
\end{thm*}
\begin{example}

Exercise 19 : \\	
Let $\mathrm{W}_1$ and $\mathrm{W}_2$ be subspaces of a vector space $\mathrm{V}.$ Prove that $\mathrm{W}_1 \cup \mathrm{W}_2 $ is a subspace of $\mathrm{W}$ if and only if $\mathrm{W}_1 \subseteq \mathrm{W}_2$ or $\mathrm{W}_2 \subseteq \mathrm{W}_1.$
\begin{sol*}$ $
\begin{enumerate}
	\item  Let $\mathrm{W}_{1}$ and  $\mathrm{W}_{2}$ be subspace of $\mathrm{W}$, $\mathrm{W}_{1}$ $\cup$ $\mathrm{W}_{2}$ is a subspace 
\item Suppose  $\mathrm{W}_{1}$ $\cup$ $\mathrm{W}_{2}$ is a subspace of $\mathrm{W}$ while $\mathrm{W}_{1} \not\subseteq \mathrm{W}_{2}$ and $\mathrm{W}_{2} \not\subseteq \mathrm{W}_{1}$ 

$\implies$ $\exists~x\in\mathrm{W}_1\,,\,x \notin \mathrm{W}_2$ , $\exists~y \in \mathrm{W}_2 \,,\, y \notin \mathrm{W}_1$ $\\$
If $\mathrm{W}_1 \cup \mathrm{W}_2$ is a subspace , $ x , y \in \mathrm{W}_1 \cup \mathrm{W}_2 $ $\implies x + y \in \mathrm{W}_1 \cup \mathrm{W}_2 $
\item $x + y \in \mathrm{W}_1 \cup \mathrm{W}_2 $ \ \ $\therefore$ $x + y \in \mathrm{W}_1$  or  $ x + y \in \mathrm{W}_2 $ \\
 Since $\mathrm{W}_1$ and $\mathrm{W}_2$ are subspaces of $\mathrm{V}$ $\therefore$ $ x \in \mathrm{W}_1,$  $y \in \mathrm{W}_2$ or $ x \in \mathrm{W}_2,$  $y \in \mathrm{W}_1$  $\rightarrow \leftarrow$ wtih our suppose in 1.
\end{enumerate} 

\end{sol*}
\end{example}


%exercise 20

\begin{example}

Exercise 20: \\	
Let Prove that if $\mathrm{W}$ is a subspace of a vector space $\mathrm{V}$ and $w_1,w_2,\dots,w_n$ are in $\mathrm{W}$, then $a_1w_1+a_2w_2+\dots +a_nw_n \in \mathrm{W}$ for any scalars $a_1,a_2,\dots , a_n$.

\begin{sol*}$ $
\begin{enumerate}
	
	\item  $\mathrm{W}$ is a subspace of $\mathrm{V} \\ $
	$\implies$ $a_1w_1,a_2w_2,\dots,a_nw_n \in \mathrm{W}$ for $a_1,a_2,\dots ,a_n \in \mathrm{F}$
	
	\item  
	\item[(i)]$a_1w_1 + a_2w_2 \in \mathrm{W}$
	\item[(ii)] Suppose $a_1w_1+a_2w_2+\dots +a_{n-1}w_{n-1} \in \mathrm{W}$
	\item[(iii)] $(a_1w_1+a_2w_2+\dots +a_{n-1}w_{n-1})+a_nw_n$ \\
				$= a_1w_1+a_2w_2+\dots +a_{n}w_{n} \in \mathrm{W}$ 

\end{enumerate} 

\end{sol*}
\end{example}



\begin{defn}
$\\$	A vector space $\mathrm{V}$ is called direct sum of $\mathrm{W}_1$ and $\mathrm{W}_2$ if $\mathrm{W}_1$ and $\mathrm{W}_2$ are subspace of $\mathrm{V}$ such thay $\mathrm{W}_1\,\cap \,\mathrm{W}_2 = \{0\}$ and $\mathrm{W}_1 + \mathrm{W}_2 = \mathrm{V}$. We denote that $\mathrm{V}$ is the direct sum of $\mathrm{W}_1$ and $\mathrm{W}_2$ by  Writting $\mathrm{V} = \mathrm{W}_1\oplus \mathrm{W}_2$.
\end{defn}

% Exercise 23

\begin{example}

Exercise 23: \\	
Let $\mathrm{W}_1 \text{and} \mathrm{W}_2$ be subspaces of a vector space $\mathrm{V}$.
\begin{enumerate}
	\item[(a)] Prove that $\mathrm{W}_1 + \mathrm{W}_2$ is a subspace of $\mathrm{V}$ that contains both $\mathrm{W}_1$ and $\mathrm{W}_2$.
	\item[(b)] Prove that any subspace of $\mathrm{V}$ that contains both $\mathrm{W}_1$ and $\mathrm{W}_2$ must also contain $\mathrm{W}_1 + \mathrm{W}_2$.

\end{enumerate}

\begin{sol*}$ $
\begin{enumerate}
	
	\item  Hello LaTeX

\end{enumerate} 

\end{sol*}
\end{example}


\subsection*{\S 1-4 Linear Combination and System of Linear Equations}
\begin{defn}[ Definition of Linear Combination ]
 $\\$	Let $\mathrm{W}$ be a vector space and $\mathrm{S}$ a nonempty subset of $\mathrm{V}.$ A vector $v \in \mathrm{V}$ is called a linear combination of vectors of $\mathrm{S}$ if there exist a finite number of vectors $u_1,u_2,\cdots,u_n$ in $\mathrm{S}$ and scalars $a_1,a_2,\cdots,a_n$ in $\mathrm{F}$ such that $v = a_1u_1 + a_2u_2 +\cdots+a_nu_n$. In this case we also say that $\mathrm{W}$ is a linear combination of $u_1,u_2,\cdots,u_n$ and call $a_1,a_2,\cdots,a_n$ the coefficients of the linear combination.
\end{defn}

% Textbook P31 Example4

\begin{defn}[ Definition of Span ]
$\\$	Let $\mathrm{S}$ be a nonempty subset of a vector space $\mathrm{V}.$ The span of $\mathrm{S},$ denoted span($\mathrm{S}$), is the set consisting of all linear combinations of the vectors in $\mathrm{S}$. For convenience, we define span($\phi$) = $\{0\}.$
\end{defn}

\begin{example}
Question ( Exercise 5 (a) (e) ) :\\
	In each part, determine whether the given vector is in the span of $S$.
\begin{enumerate}
	
\item[a.](2,-1,1), $\mathrm{S}$ = \{(1,0,2),(-1,1,1)\}
\item[e.] $-x^3+2x^2+3x+3$, $\mathrm{S}$ = $\left\{x^3+x^2+x+1,x^2+x+1,x+1\right\}$
\end{enumerate}
\begin{sol*}$ $
\begin{enumerate}
	\item[(a)]
	\item $\mathrm{Span}(S) = \{a_1(1,0,2) + a_2(-1,1,1) \mid a_1 , a_2 \in \mathrm{F}\} $
	\item $\left\{\begin{aligned}[l]
		a_1 - a_2   & = & 2\\
		0a_1 + a_2. & = &-1\\
		2a_1 + a_2  & =  &1
	\end{aligned}\right.$  $ \implies a_1 = 1, a_2 = -1$
	\item Since there exist scalars $ a_1 , a_2$ such that $(2,-1,1) = a_1(1,0,2) + a_2(-1,1,1)$, we conclude that $(2,-1,1)$ is in the span of $S$.
\end{enumerate}
\begin{enumerate}
	\item[(e)]
	\item $\mathrm{Span}(S) = \{a_1(x^3+x^2+x+1) + a_2(x^2+x+1) + a_3(x+1)\}$
	\item $\left\{\begin{aligned}[l]
		a_1 + 0a_2 + 0a_3 &=& -1\\
		a_1 + a_2 + 0a_3 &=& 2\\
		a_1 + a_2 + a_3 &=& 3\\
		a_1 + a_2 + a_3 &=& 3
	\end{aligned}\right.$ $\implies a_1 = -1 , a_2 = 3 , a_3 = 1$
	\item  Since there exist scalars $a_1 , a_2, a_ 3$ such that $-x^3+2x^2+3x+3 = a_1(x^3+x^2+x+1) + a_2(x^2+x+1) + a_3(x + 1)$, we conclude that $(2,-1,1)$ is in the span of $S$.

\end{enumerate}

\end{sol*}

\end{example}

\begin{example}
	Question ( Exercise 6 ) :\\
	Show that the vectors $(1,1,0) , (1,0,1)$, and $(0,1,1)$ generate $\mathrm{F}^3$.
\begin{sol*}
Let  $\mathrm{S} = \{ (1,0,1),(1,0,1),(0,1,1) \} $. 

 Claim : $\mathrm{S} = \{ (1,0,1),(1,0,1),(0,1,1) \} $ is linearly independent.
\begin{enumerate}
	\item $a_1(1,0,1) + a_2(1,0,1) + a_3(0,1,1) = 0 \implies a_1 = a_2 = a_3 = 0$
	\item $b_1(1,0,1) + b_2(1,0,1) + b_3(0,1,1) = (x , y , z)$
 
	$\implies b_1 = \frac{x + y - z}{2} , b_2 = \frac{x - y + z}{2} , b_3 = \frac{-x + y + z}{2}$
\end{enumerate}	
Since $\mathrm{S} = \{ (1,0,1),(1,0,1),(0,1,1) \} $ is linearly independent,  $\mathrm{Span}(S)$ can generate $\mathrm{F}^3$.
\end{sol*}

	
\end{example}

\begin{example}
	$\mathrm{S} = \{ (0,0,1),(1,0,0)\} $
	$\\$ $Span(S) = \{a_1(0,0,1) + a_2(1,0,0) \mid a_1,a_2 \in \mathrm{F} \}$  
\end{example}

\begin{thm*}[1.5]
$\\$	 The span of any subset $\mathrm{S}$ of a vector space $\mathrm{W}$ is a subspace of $\mathrm{V}.$ Moreover, any subspace of $\mathrm{W}$ that contains $\mathrm{S}$ must also contain the span of $\mathrm{S}.$
\end{thm*}

%exercise 12 , 13

\begin{example}
Exercise 12:\\
Show that a subset $\mathrm{W}$ of a vector space $\mathrm{V}$ is a subspace of $\mathrm{V}$ if and only if span$(\mathrm{W}) = \mathrm{W}$
\begin{sol*} $ $\\
	\begin{enumerate}
		\item Hello LaTeX
	\end{enumerate}


\end{sol*}
\end{example}


\begin{example}
Exercise 13:\\
Show that if $\mathrm{S}_1 \text{ and } \mathrm{S}_2$ are subsets of a vector space $\mathrm{V}$ such that $\mathrm{S}_1 \subseteq \mathrm{S}_2$, then $\text{span}(\mathrm{S}_1) \subseteq \text{span}(\mathrm{S}_2)$. In particular, if $\mathrm{S}_1 \subseteq \mathrm{S}_2$ and span($\mathrm{S}_1$)$ = \mathrm{V}$, deduce that span$(\mathrm{S}_2) = \mathrm{V}$
\begin{sol*} $ $\\
	\begin{enumerate}
		\item Hello LaTeX
	\end{enumerate}


\end{sol*}
\end{example}

\subsection*{\S 1-5 Linear dependence and linear independence}
\begin{defn}[ Linearly Dependent ]
$\\$	A subset $\mathrm{S}$ of a vector space $\mathrm{W}$ is called linearly dependent if there exist a finite number of distinct vectors $u_1, u_2, . . . , u_n$ in $\mathrm{S}$ and scalars $a_1,a_2,...,a_n$, not all zero, such that 

\end{defn}

\begin{center}
	$a_1u_1 + a_2u_2 +\cdots +a_nu_n = 0$.
\end{center}
In this case we also say that the vectors of $\mathrm{S}$ are linearly dependent.


\begin{defn}[ Definition of Linearly Independent ]
$\\$	A subset $\mathrm{S}$ of a vector space that is not linearly dependent is called linearly independent. As before, we also say that the vectors of $\mathrm{S}$ are linearly independent.

\end{defn}
%exercise 2 (c) (e)



\begin{thm*}[1.6]
$\\$	Let $\mathrm{W}$ be a vector space,and let $\mathrm{S}_1 \subseteq \mathrm{S}_2 \subseteq \mathrm{V}.$ If $\mathrm{S}_1$ is linearly dependent, then $\mathrm{S}_2$ is linearly dependent.
$\\$Corollary. Let $\mathrm{W}$ be a vector space, and let $\mathrm{S}_1 \subseteq \mathrm{S}_2 \subseteq \mathrm{V}.$ If $\mathrm{S}_2$ is linearly independent, then $\mathrm{S}_1$ is linearly independent.
\end{thm*}



% Exercise 2 ACE, 13

\begin{example}
Exercise 2:\\
Determine whether the following sets are linearly dependent or linearly independent.

	\begin{enumerate}
		\item[(c)] $\{x^3+2x^2,-x^2+3x+1,x^3-x^2+2x-1\}$ in $\mathrm{P}_3(\mathbb{R})$
		\item[(e)] $\{ (1,-1,2),(1,-2,1),(1,1,4)\}$ in $\mathbb{R}^3$
	\end{enumerate}

\begin{sol*} $ $\\
	\begin{enumerate}
		\item Hello LaTeX
	\end{enumerate}


\end{sol*}
\end{example}

\begin{example}
	Question ( Exercise 2 (a) (c) (e) ) :\\
	Determine whether the following sets are linearly dependent or linearly independent.
	 \begin{enumerate}
	 	\item[a.] $\left\{\left[\begin{matrix} 
1  & -3\\
-2 & 4

 \end{matrix}\right] , \left[\begin{matrix} 
-2 & 6 \\
4 & -8
 \end{matrix}\right]\right\}$ 
 \item[c.]$\left\{x^3+2x^2 , -x^2+3x+1 , x^3 - x^2 + 2x +1\right\}$
 \item[e.]$\left\{(1,-1,2) , (1,-2,1) , (1,1,4)\right\}$ in $\R^3$
 \end{enumerate}
 \begin{sol*}$ $
	 \begin{enumerate}
 	 \item [a.] Since $ 2\left[\begin{matrix}
1 & -3 \\
2 & 4 	
\end{matrix}\right] + \left[\begin{matrix}
-2 & 6 \\
4 & 8	
\end{matrix}\right] = 0\implies$ linearly dependent by Thm 1.5. 
 
 	\end{enumerate}
 \end{sol*}
\end{example}

\begin{thm*}[1.7]
$\\$	Let $\mathrm{S}$ be a linearly independent subset of a vector space $\mathrm{V}$, and let v be a vector in V that is not in S. Then $\mathrm{S}\cup \{ v \}$ is linearly dependent if and only if v $\in$ span(S).

\end{thm*}

\subsection*{\S 1-6 Bases and Dimension}
\begin{defn}[ Definition of Bases ]	
$\\$	 A basis $\beta$ for a vector space V is a linearly independent subset of V that generates V. If $\beta$ is a basis for V, we also say that the vectors of $\beta$ form a basis for V.
\end{defn}



% Exercise 2 (b) , 3(a) ,13


\begin{thm*}[1.8]
$\\$	 Let $\mathrm{V}$ be a vector space and $\beta = \{u_1,u_2,...,u_n \}$ be a subset of $\mathrm{V}$. Then $\beta$ is a basis for V if and only if each  $v\in$ V can be uniquely expressed as a linear combination of vectors of $\beta$, that is, can be expressed in the form
\begin{center}
	$\mathrm{v} = a_1u_1 + a_2u_2 + \cdots+ a_nu_n $
\end{center} 
for unique scalars $a_1, a_2, \cdots , a_n$.
\end{thm*}
%exercise 19
 
\begin{thm*}[1.9]
$\\$ If a vector space V is generated by a finite set S, then some subset of S is a basis for V. Hence V has a finite basis.
\end{thm*}

\begin{thm*}[1.10 Replacement Theorem]
$\\$ Let V be a vector space that is generated by a set G containing exactly n vectors, and let L be a linearly independent subset of V containing exactly m vectors. Then m $\leq$ n and there exists a subset H of G containing exactly $\mathrm{n} - \mathrm{m}$ vectors such that L $\cup$ H generates V.	
\end{thm*}

\begin{thm*}[Corollary 1.]
	$\\$ Let V be a vector space having a finite basis. Then every basis for V contains the same number of vectors.
\end{thm*}


\begin{defn}[ Definition of Dimension ]	
$\\$ A vector space is called finite-dimensional if it has a basis consisting of a finite number of vectors. The unique number of vectors
 in each basis for V is called the dimension of V and is denoted by $dim(\mathrm{V})$.
A vector space that is not finite-dimensional is called infinite-dimensional	.
\end{defn}

% Exercise 14, 19,22 ,26 , 29

\begin{example}
Exercise 14:\\
Find bases for the following subspaces of $\mathbb{F}^5$:\\
\begin{center}
	$\mathrm{W}_1 = \{(a_1,a_2,a_3,a_4,a_5)\ \in \mathbb{F}^5 : a_1 -a_3 - a_4=0\}$
\end{center}
and\\
\begin{center}
	$\mathrm{W}_2 = \{(a_1,a_2,a_3,a_4,a_5) \in \mathrm{F}^5 : a_2 = a_3 = a_4$ and $a_1 + a_5 = 0\}$.
\end{center}
What are the dimensions of $\mathrm{W}_1$ and $\mathrm{W}_2$

\begin{sol*} $ $\\
	\begin{enumerate}
		\item Hello LaTeX
	\end{enumerate}


\end{sol*}
\end{example}

\begin{example}
Exercise 19:\\
the converse side proof of Thm. 1.8\\
\begin{sol*} $ $\\
	\begin{enumerate}
		\item Hello LaTeX
	\end{enumerate}


\end{sol*}
\end{example}

\begin{example}
Exercise 26:\\
For a fixed $a \in \mathbb{R}$, determine the dimension of the subspace of $\mathrm{P}_n(\mathbb{R})$ define by \{$ f \in \mathrm{P}_n(\mathbb{R}):f(a)=0$\}
\begin{sol*} $ $\\
	\begin{enumerate}
		\item Hello LaTeX
	\end{enumerate}


\end{sol*}
\end{example}

\begin{example}
Exercise 29:\\
\begin{enumerate}
	\item[(a)] Prove that if $\mathrm{W}_1$ and $\mathrm{W}_2$ are finite-dimensional subspaces of a vector space $\mathrm{V}$, then the subspace $\mathrm{W}_1 + \mathrm{W}_2$ is finite-dimensional, and dim($\mathrm{W}_1 +\mathrm{W}_2)$ = dim($\mathrm{W}_1$)+dim($\mathrm{W}_2)-$ dim($\mathrm{W}_1 \cap \mathrm{W}_2$).
	\item[(b)] Let $\mathrm{W}_1$ and $\mathrm{W}_2$ be finite-dimensional subspaces of a vector space $\mathrm{V}$, and let $\mathrm{V} = \mathrm{W}_1 + \mathrm{W}_2$. Deduce that $\mathrm{V}$ is the direct sum of $\mathrm{W}_1$ and $\mathrm{W}_2$ if and only if dim($\mathrm{V}$) = dim($\mathrm{W}_1$) + dim($\mathrm{W}_2$).
\end{enumerate}


\begin{sol*} $ $\\
	\begin{enumerate}
		\item Hello LaTeX
	\end{enumerate}


\end{sol*}
\end{example}
 
\end{document} 