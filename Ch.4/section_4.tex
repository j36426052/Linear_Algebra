\input{settings.tex}
\linespread{1.6}
\begin{document}
	\section*{Chap.4 Determinants}
		\subsection*{\S 4-1 Determinants of order 2}
		
		\begin{defn}
			If $A = \left( \begin{matrix}a & b\\c & d
			\end{matrix} \right)$is a $2\times 2$matrix with entries from a field F, then we define the determinant of A, denoted det($A$), to be the scalar $ad - bc$.
		\end{defn}
		
		\begin{example}
			%just choose any 2x2 example%
		\end{example}
		
		\subsection*{\S 4-2 Determinants of order n}
		
		\begin{defn}
			Let $A \in M_{n\times n}(F)$. If $n=1$, so that $A = (A_{11})$, we define det$(A)=A_{11}$.For $n \geq 2$, we define det$(A)$ recursively as
		
			\[\rm	 det(A)=\sum_{j=1}^{n}(-1)^{1+j} A_{1j} \cdot  det(\widetilde{A}_{1j} )    .\]
		
		The scalar 
		\[\rm	 (-1)^{1+j}   det(\widetilde{A}_{1j} )    .\]
		is called the cofactor of the entry of $A$ in row $i$,column $j$.
		Letting 
		\[\rm	 c_{ij} = (-1)^{i+j}det(\widetilde{A}_{ij})    .\]
		
		denote the cofactor of the row $i$, column $j$ entry of $A$, we can express the formula for the determinant of $A$ as 
		
		\[\rm	 det(A)=A_{11}c_{11} + A_{12}c_{12} + \cdots + A_{1n}c_{1n}    .\]
		\end{defn}
		
				
		\begin{thm*}
		$ $\\ The determinant of a square matrix can be evaluated by
cofactor expansion along any row. That is, if $A \in M_{n\times n}(F)$, then for any integer $i(1\leq i\leq n)$,

			\[\rm	 det(A)=\sum_{j=1}^{n}(-1)^{i+j} A_{ij} \cdot  det(\widetilde{A}_{ij} )    .\]
		\end{thm*}
		
		\begin{example}
			%just choose any nxn example%
		\end{example}
		
		
		\begin{thm*}
		$ $\\The determinant of an $n \times n$ matrix is a linear function of each row when the remaining rows are held fixed. That is, for $1 \leq r \leq n$, we have
		
		\[\rm	 
		det\left( \begin{matrix} a_1 \\ \vdots \\ a_{r-1} \\ u + kv \\ a_{r+1} \\ \vdots \\ a_n \end{matrix}\right) = 
		det\left( \begin{matrix} a_1 \\ \vdots \\ a_{r-1} \\ u \\ a_{r+1} \\ \vdots \\ a_n \end{matrix}\right) + 
		kdet\left( \begin{matrix} a_1 \\ \vdots \\ a_{r-1} \\ v \\ a_{r+1} \\ \vdots \\ a_n \end{matrix}\right)   .\]
		
		whenever k is a scalar and $u, v,$ and each $a_i$ are row vectors in $F^n$
		
		\end{thm*}
		
		\begin{thm*}
		$ $\\Let $A \in M_{n \times n}(F)$, and let $B$ be a matrix obtained by adding a multiple of one row of $A$ to another row of $A$. Then $det(B) = det(A)$.
		\end{thm*}
		
		The following rules summarize the effect of an elementary row operation on the determinant of a matrix $A \in M_{n \times n}(F)$.
		
		\begin{enumerate} 
			\item[(a)] If $B$ is a matrix obtained by interchanging any two rows of $A$, then $det(B) = − det(A)$.
			\item[(b)] If $B$ is a matrix obtained by multiplying a row of $A$ by a nonzero scalar $k$, then $det(B) = k det(A)$.
			\item[(c)] If $B$ is a matrix obtained by adding a multiple of one row of $A$ to another row of $A$, then $det(B) = det(A)$.
		\end{enumerate}
		
				
		\begin{example}
			%choose a question which can be simple by the ERO%
		\end{example}
		
		\subsection*{\S 4-3 Properties of determinants}
		
		\begin{thm*}
			$ $\\
			For any $A,B \in M_{n \times n}(F), det(AB)=det(A)\cdot det(B)$.
		\end{thm*}
		
		Corollary. A matrix $A \in M_{n \times n}(F)$ is invertible if and only if $det(A) \neq 0$. Furthermore, if $A$ is invertible, then $det(A^{-1})=\frac{1}{det(A)}$
		
		\begin{thm*}
			$ $\\ For any $A \in M_{n \times n}(F), det(A^t)=det(A)$ 
		\end{thm*}
		
		
		%other exercise ~ (for example the proof of det() of orthogonal)%
\end{document}