%% LyX 2.3.5.2 created this file.  For more info, see http://www.lyx.org/.
%% Do not edit unless you really know what you are doing.
\documentclass[12pt,english]{article}
\usepackage{fontspec}
	\setmainfont{Cochin}
\usepackage{setspace}
	\onehalfspace
\usepackage{amssymb, amsmath, amsthm}
\usepackage{mathrsfs}
\usepackage{tasks}
\renewcommand{\ttdefault}{cmtl}
\usepackage{tcolorbox}
\usepackage[framemethod=tikz]{mdframed} % Allows defining custom boxed/framed environments
\usepackage[a4paper]{geometry}
\geometry{verbose,tmargin=2.5cm,bmargin=2.5cm,lmargin=3cm,rmargin=3cm}
\usepackage{enumitem}
\everymath{\displaystyle}

\makeatletter
%%%%%%%%%%%%%%%%%%%%%%%%%%%%%% Textclass specific LaTeX commands.
\theoremstyle{plain}
\newtheorem{thm}{\protect\theoremname}[section]
\newtheorem{cor}[thm]{Corollary}
\newtheorem{lmma}[thm]{Lemma}
\newtheorem*{defn}{\underline{Definition}}
\newtheorem*{ex*}{Example}
\newtheorem*{sol*}{Solution}
\newtheorem*{thm*}{Theorem}
\newtheorem*{lmma*}{Lemma}
\newtheorem*{rmk*}{Remark}
\newtheorem*{pf*}{\underline{\textbf{Proof\ }}}
%%%%%%%%%%%%%%%%%%%%%%%%%%%%%% User specified LaTeX commands.
\renewcommand{\P}{\mathscr{P}}
\newcommand{\B}{\mathscr{B}}
\newcommand{\A}{\mathscr{A}}
\newcommand{\C}{\mathbb{C}}
\newcommand{\CC}{\mathscr{C}}
\newcommand{\R}{\mathbb{R}}
\newcommand{\Q}{\mathbb{Q}}
\newcommand{\Z}{\mathbb{Z}}
\newcommand{\N}{\mathbb{N}}
\newcommand{\X}{\mathcal{X}}
\newcommand{\T}{\mathscr{T}}
\newcommand{\arbuni}{\bigcup_{\alpha\in I}}
\newcommand{\finint}{\bigcap_{i=1}^n}
\newcommand{\Ua}{{\textsc{U}_\alpha}}
\newcommand{\Ui}{\textsc{U}_i}

\newcommand{\pair}[2]{\left( \,#1\,,\,#2\,\right) }
\makeatother

\usepackage{babel}
\providecommand{\definitionname}{Definition}
\providecommand{\theoremname}{Theorem}
%----------------------------------------------------------------------------------------
%	INFORMATION ENVIRONMENT
%----------------------------------------------------------------------------------------

% Usage:
% \begin{info}[optional title, defaults to "Info:"]
% 	contents
% 	\end{info}

\mdfdefinestyle{info}{%
	topline=false, bottomline=false,
	leftline=false, rightline=false,
	nobreak,
	singleextra={%
		\fill[black](P-|O)circle[radius=0.4em];
		\node at(P-|O){\color{white}\scriptsize\bf i};
		\draw[very thick](P-|O)++(0,-0.8em)--(O);%--(O-|P);
	}
}

% Define a custom environment for information
\newenvironment{info}[1][Info:]{ % Set the default title to "Info:"
	\medskip
	\begin{mdframed}[style=info]
		\noindent{\textbf{#1}}
}{
	\end{mdframed}
}
%----------------------------------------------------------------------------------------
%	WARNING TEXT ENVIRONMENT
%----------------------------------------------------------------------------------------

% Usage:
% \begin{warn}[optional title, defaults to "Warning:"]
%	Contents
% \end{warn}

\mdfdefinestyle{warning}{
	topline=false, bottomline=false,
	leftline=false, rightline=false,
	nobreak,
	singleextra={%
		\draw(P-|O)++(-0.5em,0)node(tmp1){};
		\draw(P-|O)++(0.5em,0)node(tmp2){};
		\fill[black,rotate around={45:(P-|O)}](tmp1)rectangle(tmp2);
		\node at(P-|O){\color{white}\scriptsize\bf !};
		\draw[very thick](P-|O)++(0,-1em)--(O);%--(O-|P);
	}
}

% Define a custom environment for warning text
\newenvironment{warn}[1][Warning:]{ % Set the default warning to "Warning:"
	\medskip
	\begin{mdframed}[style=warning]
		\noindent{\textbf{#1}}
}{
	\end{mdframed}
}
%----------------------------------------------------------------------------------------
%	NUMBERED QUESTIONS ENVIRONMENT
%----------------------------------------------------------------------------------------

% Usage:
% \begin{question}[optional title]
%	Question contents
% \end{question}

\mdfdefinestyle{Example}{
	innertopmargin=1.2\baselineskip,
	innerbottommargin=0.8\baselineskip,
	roundcorner=5pt,
	nobreak,
	singleextra={%
		\draw(P-|O)node[xshift=1em,anchor=west,fill=white,draw,rounded corners=5pt]{%
		Example \exampleTitle};
	}
}

%\newcounter{Question} % Stores the current question number that gets iterated with each new question

% Define a custom environment for numbered questions
\newenvironment{example}[1][\unskip]{
	\bigskip
	%\stepcounter{Question}
	\newcommand{\exampleTitle}{~#1}
	\begin{mdframed}[style=Example]
}{
	\end{mdframed}
	\medskip
}








% Begin from here




\begin{document}
\title{CH1 Vector Space Textbook}
\maketitle

\section*{Vector Space}

\begin{defn}[ Vector Space ]
$\\$	A vector space (or linear space) $\mathrm{W}$ over a $field^2$ $\mathrm{F}$ consists of a set on which two operations (called addition and scalar mul- tiplication, respectively) are defined so that for each pair of elements $x, y$, in $\mathrm{W}$ there is a unique element $x+y$ in $\mathrm{W}$, and for each element a in F and each element $x$ in $\mathrm{W}$ there is a unique element $ax$ in $\mathrm{W}$, such that the following conditions hold.

\end{defn}
\subsection*{\color{red}There are eight rules for calculating the vector space !\\Please do not forget !\\}

\begin{defn}[ Definition of Subspace ]
$\\$
	A subset $\mathrm{W}$ of a vector space $\mathrm{W}$ over a field $\mathrm{F}$ is called a subspace of $\mathrm{W}$ if $\mathrm{W}$ is a vector space over $\mathrm{F}$ with the operations of addition and scalar multiplication defined on $\mathrm{W}$.\\

\end{defn}


\begin{thm*}[1.3] 
$\\$ Let V be a vector space and W a subset of V. Then W is a subspace of V if and only if the following three conditions hold for the operations defined in V.
\\
(a)\ \ $0 \in \mathrm{W}.$ \\
(b)\ \ $x+y \in W$ whenever $x \in \mathrm{W}$ and $y \in \mathrm{W}.$ \\
(c)\ \ $cx \in \mathrm{W}$ whenever $c \in \mathrm{F}$ and $x \in \mathrm{W}.$ \\

\end{thm*}

% Exercise 8:a,c,f


\begin{thm*}[1.4]
$\\$ Any intersection of subspaces of a vector space $\mathrm{W}$ is a subspace of $\mathrm{V}$.
	
\end{thm*}

% Exercise 19
\begin{defn}
$\\$	A vector space $\mathrm{V}$ is called direct sum of $\mathrm{W}_1$ and $\mathrm{W}_2$ if $\mathrm{W}_1$ and $\mathrm{W}_2$ are subspace of $\mathrm{V}$ such thay $\mathrm{W}_1\,\cap \,\mathrm{W}_2 = \{0\}$ and $\mathrm{W}_1 + \mathrm{W}_2 = \mathrm{V}$. We denote that $\mathrm{V}$ is the direct sum of $\mathrm{W}_1$ and $\mathrm{W}_2$ by  Writting $\mathrm{V} = \mathrm{W}_1\oplus \mathrm{W}_2$.
\end{defn}
% Exercise 23

\begin{example}

Question ( Exercise 19 ) : \\	
Let $\mathrm{W}_1$ and $\mathrm{W}_2$ be subspaces of a vector space $\mathrm{V}.$ Prove that $\mathrm{W}_1 \cup \mathrm{W}_2 $ is a subspace of $\mathrm{W}$ if and only if $\mathrm{W}_1 \subseteq \mathrm{W}_2$ or $\mathrm{W}_2 \subseteq \mathrm{W}_1.$
\begin{sol*}$ $
\begin{enumerate}
	\item  Let $\mathrm{W}_{1}$ and  $\mathrm{W}_{2}$ be subspace of $\mathrm{W}$, $\mathrm{W}_{1}$ $\cup$ $\mathrm{W}_{2}$ is a subspace 
\item Suppose  $\mathrm{W}_{1}$ $\cup$ $\mathrm{W}_{2}$ is a subspace of $\mathrm{W}$ while $\mathrm{W}_{1} \not\subseteq \mathrm{W}_{2}$ and $\mathrm{W}_{2} \not\subseteq \mathrm{W}_{1}$ 
$\rightarrow$ $\exists \ \ x \in \mathrm{W}_1, x\notin \mathrm{W}_2$ , $\exists \ \ y \in \mathrm{W}_2, y \notin \mathrm{W}_1$ $\\$
If $\mathrm{W}_1 \cup \mathrm{W}_2$ is a subspace , $ x , y \in \mathrm{W}_1 \cup \mathrm{W}_2 $ $\rightarrow  x + y \in \mathrm{W}_1 \cup \mathrm{W}_2 $
\item $x + y \in \mathrm{W}_1 \cup \mathrm{W}_2 $ \ \ $\therefore$ $x + y \in \mathrm{W}_1$  or  $ x + y \in \mathrm{W}_2 $ \\
 Since $\mathrm{W}_1$ and $\mathrm{W}_2$ are subspaces of $\mathrm{V}$ \ \ $\therefore$ $ x \in \mathrm{W}_1,$  $y \in \mathrm{W}_2$ or $ x \in \mathrm{W}_2,$  $y \in \mathrm{W}_1$  $\rightarrow \leftarrow$ wtih (1)
\end{enumerate} 

\end{sol*}
\end{example}


\begin{defn}[ Definition of Linear Combination ]
 $\\$	Let $\mathrm{W}$ be a vector space and $\mathrm{S}$ a nonempty subset of $\mathrm{V}.$ A vector $v \in \mathrm{V}$ is called a linear combination of vectors of $\mathrm{S}$ if there exist a finite number of vectors $u_1,u_2,...,u_n$ in $\mathrm{S}$ and scalars $a_1,a_2,...,a_n$ in $\mathrm{F}$ such that $v = a_1u_1 + a_2u_2 +���+a_nu_n.$ In this case we also say that $\mathrm{W}$ is a linear combination of $u_1,u_2,...,u_n$ and call $a_1,a_2,...,a_n$ the coefficients of the linear combination.
\end{defn}

% Textbook P31 Example4

\begin{defn}[ Definition of Span ]
$\\$	Let $\mathrm{S}$ be a nonempty subset of a vector space $\mathrm{V}.$ The span of $\mathrm{S},$ denoted span($\mathrm{S}$), is the set consisting of all linear combinations of the vectors in $\mathrm{S}$. For convenience, we define span($\phi$) = $\{0\}.$
\end{defn}

% Textbook P34 Exercise 5 AE,6 

\begin{example}
	$\mathrm{S} = \{ (0,0,1),(1,0,0)\} $
	$\\$ $Span(S) = \{a_1(0,0,1) + a_2(1,0,0) \mid a_1,a_2 \in \mathrm{F} \}$  
\end{example}

\begin{thm*}[1.5]
$\\$	 The span of any subset $\mathrm{S}$ of a vector space $\mathrm{W}$ is a subspace of $\mathrm{V}.$ Moreover, any subspace of $\mathrm{W}$ that contains $\mathrm{S}$ must also contain the span of $\mathrm{S}.$
\end{thm*}


\begin{defn}[ Linearly Dependent ]
$\\$	A subset $\mathrm{S}$ of a vector space $\mathrm{W}$ is called linearly dependent if there exist a finite number of distinct vectors $u_1, u_2, . . . , u_n$ in $\mathrm{S}$ and scalars $a_1,a_2,...,a_n$, not all zero, such that 

\end{defn}

\begin{center}
	$a_1u_1 + a_2u_2 +���+a_nu_n = 0.$
\end{center}
In this case we also say that the vectors of $\mathrm{S}$ are linearly dependent.


\begin{defn}[ Definition of Linearly Independent ]
$\\$	A subset $\mathrm{S}$ of a vector space that is not linearly dependent is called linearly independent. As before, we also say that the vectors of $\mathrm{S}$ are linearly independent.

\end{defn}


\begin{thm*}[1.6]
$\\$	Let $\mathrm{W}$ be a vector space,and let $\mathrm{S}_1 \subseteq \mathrm{S}_2 \subseteq \mathrm{V}.$ If $\mathrm{S}_1$ is linearly dependent, then $\mathrm{S}_2$ is linearly dependent.
$\\$Corollary. Let $\mathrm{W}$ be a vector space, and let $\mathrm{S}_1 \subseteq \mathrm{S}_2 \subseteq \mathrm{V}.$ If $\mathrm{S}_2$ is linearly independent, then $\mathrm{S}_1$ is linearly independent.
\end{thm*}



% Exercise 2 ACE, 13


% Thm 1.7
\begin{thm*}[1.7]
$\\$	Let $\mathrm{S}$ be a linearly independent subset of a vector space $\mathrm{V}$, and let v be a vector in V that is not in S. Then $\mathrm{S}\cup \{ v \}$ is linearly dependent if and only if v $\in$ span(S).

\end{thm*}

 
% Definition of Bases
\begin{defn}[ Definition of Bases ]	
$\\$	 A basis $\beta$ for a vector space V is a linearly independent subset of V that generates V. If $\beta$ is a basis for V, we also say that the vectors of $\beta$ form a basis for V.
\end{defn}



% Exercise 2 BD


% Thm 1.8
\begin{thm*}[1.8]
$\\$	 Let $\mathrm{V}$ be a vector space and $\beta = \{u_1,u_2,...,u_n \}$ be a subset of $\mathrm{V}$. Then $\beta$ is a basis for V if and only if each v $\in$ V can be uniquely expressed as a linear combination of vectors of $\beta$, that is, can be expressed in the form
\begin{center}
	$\mathrm{v} = a_1u_1 + a_2u_2 + � � � + a_nu_n $
\end{center} 
for unique scalars $a_1, a_2, . . . , a_n$.
\end{thm*}


% Thm 1.9 
\begin{thm*}[1.9]
$\\$ If a vector space V is generated by a finite set S, then some subset of S is a basis for V. Hence V has a finite basis.
\end{thm*}

% Thm 1.10
\begin{thm*}[1.10 Replacement Theorem]
$\\$ Let V be a vector space that is generated by a set G containing exactly n vectors, and let L be a linearly independent subset of V containing exactly m vectors. Then m $\leq$ n and there exists a subset H of G containing exactly $\mathrm{n} - \mathrm{m}$ vectors such that L $\cup$ H generates V.	
\end{thm*}
% Corollary 
\begin{thm*}[Corollary 1.]
	$\\$ Let V be a vector space having a finite basis. Then every basis for V contains the same number of vectors.
\end{thm*}



% Definition of Dimension
\begin{defn}[ Definition of Dimension ]	
$\\$ A vector space is called finite-dimensional if it has a basis consisting of a finite number of vectors. The unique number of vectors
 in each basis for V is called the dimension of V and is denoted by $dim(\mathrm{V})$.
A vector space that is not finite-dimensional is called infinite-dimensional	.
\end{defn}

% Exercise 14, 22, 29

%Trivial




% Chapter 2 begins from here


\newpage
\section*{Chapter 2}	
	\subsection*{\S\,2-1 Linear transformation}
\begin{defn}[Definition of Linear Transformation]
$ $\\	Let $\mathrm{V}$ and $\mathrm{W}$ be vector spaces (over $\mathrm{F}$). We call a function $\mathrm{T}: \mathrm{V}  \rightarrow  \mathrm{W}$    a linear transformation from $\mathrm{V}$ to $\mathrm{W}$ if, for all x, y $\in$ $\mathrm{V}$ and c $\in$ $\mathrm{F}$, we have

\begin{enumerate}
	\item [(a)]   $\mathrm{T}$(x + y) = $\mathrm{T}$(x) + $\mathrm{T}$(y) 
	\item [(b)]    $\mathrm{T}$($\mathrm{c}$x) = $\mathrm{c}\mathrm{T}$(x)
\end{enumerate} 
\end{defn}

\begin{defn}[Definition of nullity and rank]
$ $\\	Let $\mathrm{V}$ and $\mathrm{W}$ be vector spaces and let $\mathrm{T}: \mathrm{V}  \rightarrow  \mathrm{W}$ be linear. If N(T) and R(T) are finite-dimensional, then we define the nullity of T, denoted nullity(T), and the rank of T, denoted rank(T), to be the dimensions of N(T) and R(T), respectively.
\end{defn}

\begin{thm*}
$ $ \\ 	Let V and W be vector spaces and $\mathrm{T}: \mathrm{V}  \rightarrow  \mathrm{W}$ be linear. Then N(T) and R(T) are subspaces of $\mathrm{V}$ and $\mathrm{W}$, respectively.
\end{thm*}

\begin{thm*}
$ $\\
 Let V and W be vector spaces, and let $\mathrm{T} : \mathrm{V}  \rightarrow \mathrm{W}$ be linear. If $\beta = \{ v_1,v_2,...,v_n \}$ is a basis for $\mathrm{V}$ , then $ \mathrm{R}(\mathrm{T}) = \mathrm{span}(\mathrm{T}(\beta)) = \mathrm{span}(\{\mathrm{T}(v_1), \mathrm{T}(v_2), . . . , \mathrm{T}(v_n)\})$.

\end{thm*}

\begin{example}[11] %Example 11
	Question : \\
	Let T : $\mathrm{P}_2(R) \rightarrow  \mathrm{P}_3(R) $ be the linear trans formation defined by 
	
	
	 \[T(f(x)) = 2f'(x) + \int_0^x3 f(t) dt \]
	
		\begin{sol*} 
$ $ \\ 	
Now $\mathrm{R}(\mathrm{T}) = span( \{ \mathrm{T}(1), \mathrm{T}(x), \mathrm{T}(x^2 )  ) = span( \{ 3x,  2+ \frac{3}{2}  x^2 , 4x + x^3 \} ) $. 

 \noindent Since \{ $3x$ , $ 2 + \frac{3}{2} x^2$, $4x + x^3$ \} is linearly independent, rank($\mathrm{T}$) = 3. Since
2
$dim(P3(R)) = 4$ , T is not onto. From the dimension theorem (Thm 2.3) , nullity(T) +
3 = 3. So nullity(T) = 0, and therefore, N(T) = {0}. We conclude from Theorem 2.4 that T is one-to-one.

	\end{sol*}

	\end{example}
\begin{example}[12] %Example 12
	 Question : \\ Let $\mathrm{T}: \mathrm{F}^2 \rightarrow \mathrm{F}^2 $ be the linear transformation defined by 
	 
	 \[T(a_1, a_2) = (a_1 + a_2, a_1)\]

	\begin{sol*} 
$ $ \\
		It is easy to see that N(T) = \{ 0 \} ; so T is one-to-one. Hence Theorem 2.5 tells us that T must be onto. 
	\end{sol*}

\end{example}

\begin{thm*}[2.3 Dimension Theorem]
$ $ \\	
Let V and W be vector spaces,
and let $ \mathrm{T} : \mathrm{V}  \rightarrow  \mathrm{W} $ be linear. \\ If V is finite-dimensional, then
 
 
 \[\rm	nullity(T) + rank(T) = dim(V).\]


\end{thm*}

\begin{thm*}[2.4]
$ $ \\	Let V and W be vector spaces, and let $\rm T: V  \rightarrow  W$ be
linear. Then T is one-to-one if and only if N(T) = {0 }.

\end{thm*}

\begin{thm*}[2.5]
$ $ \\	 Let V and W be vector spaces of equal (finite) dimension, and let $\rm T: V  \rightarrow  W$ be linear. Then the following are equivalent.
\begin{enumerate} 
	\item[(a)] T is one-to-one. 
	\item[(b)] T is onto.
	\item[(c)] rank(T) = dim(V).

\end{enumerate}

\end{thm*}

\begin{example}
	Let $\rm T : \R^2\longrightarrow \R^3 $ , $\rm T(a_1,a_2) = (a_1+a_2 , 0 , 2a_1-a_2)$ , determine that $\rm T $ is linear , 1-1 , onto or not.
	\begin{sol*}
		 $ $
		\begin{itemize}
			\item Calim : $\rm T$ is linear.\newline
			Let $ x =(a_1,a_2) \,,\, y= (b_1,b_2)$ 
				
				\(\begin{aligned} \rm T(c x+y) &=\rm T\left(c\left(a_{1}, a_{2}\right)+\left(b_{1}, b_{2}\right)\right)= T\left(ca_{1}+b_{1}, ca_{2}+b_{2}\right) \\ &=\rm \left(c a_{1}+b_{1}+c a_{2}+b_{2}, 0,2 c a_{1}+2 b_{1}-ca_{2}-b_{2}\right) \\ &=\rm \left(c\left(a_{1}+a_{2}\right)+\left(b_{1}+b_{2}\right), 0, c\left(2 a_{1}-a_{2}\right)+\left(2 b_{1}-b_{2}\right)\right) \\ &=\rm c\left(a_{1}+a_{2}, 0,2 a_{1}-a_{2}\right)+\left(b_{1}+b_{2}, 0,2 b_{1}-b_{2}\right) \\ &=\rm c T(x)+T (y) \end{aligned}\)
				
			$\therefore \rm T $ is linear
			\item 1-1 and onto 	
			
			By Thm 2.2 in text book choose a basis for $\R^2 , \beta = \{(1,0) , (0,1)\}$
			
			$\rm R(T) = span\left(T(\beta)\right) = span(\left\{T(1,0) , T(0,1)\right\}) = span\left((1,0,-1), (1,0,-2)\right)$ 
			
			Clearly it is L.I. $\rm\implies rank(T) = 2$ , and apply the Dimension Theorem $\rm rank(T)=2\neq3=\dim(\R^3) $, and $\rm nullitily(T) = 1 $ , then it is not onto.
			
			By Thm 2.4  it is not one to one.
			
		\end{itemize}
	\end{sol*}
		\end{example}
		\begin{example}
		Let \(\rm V\) and \(\rm W\) be vector spaces, let \(\mathrm{T}:\rm  V \rightarrow W\) be linear, and let
\(\left\{w_{1}, w_{2}, \ldots, w_{k}\right\}\) be a linearly independent subset of \(\rm R(T)\). Prove that
if \(S=\left\{v_{1}, v_{2}, \ldots, v_{k}\right\}\) is chosen so that \(\mathrm{T}\left(v_{i}\right)=w_{i}\) for \(i=1,2, \ldots, k\),
then \(S\) is linearly independent.
\begin{sol*}$ $
Calim : $\sum_{i=1}^n a_iv_i = 0\implies a_1 = a_2 = a_3 =\cdots =a_n = 0 $
	
	Let \(\sum_{i=1}^{n} a_{i} v_{i}=0\)
then \(\mathrm{T}\left(\sum_{i=1}^{n} a_{i} v_{i}\right)=0\)
Since \(\rm T\) is linear , \(\mathrm{T}\left(\sum_{i=1}^{n} a_{i} v_{i}\right)=\sum_{i=1}^{n} a_{i} \mathrm{T}\left(v_{i}  \right)=\sum_{i=1}^na_iw_i = 0\)
Since $\mathrm{S}$ is L.I. $\implies a_1 = a_2=\cdots = a_n = 0$ 
\end{sol*}
	

\end{example}
\begin{thm*}[2.6]
$ $ \\Let V and W be vector spaces over F, and suppose that \{ $v_1,v_2,...,v_n$ \} is a basis for V. For $w_1,w_2,...,w_n$ in W, there exists exactly one linear transformation $T: V  \rightarrow  W$ such that T($v_i$) = $w_i$ for i = 1,2,...,n.
\end{thm*}


	 
\end{document} 