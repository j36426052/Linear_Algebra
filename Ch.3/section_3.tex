\input{settings3.tex}

\begin{document}
\chapter{\S\  Ch3 Elementary Matrix Operations and Systems of Linear Equations}

Learning Goal : 
1. "rank-preserving" operations \\
2. solve the linear system 

\section{3.1 Elementary Matrix Operations and Elementary Matrices }
%definition of elementary operations 	
	
%definition of type 1, 2, 3 elementary matrices

%Fact that any elementary matrix has 2 ways to discribe

%Thm 3.1 

%Thm 3.2 with 1 example 


\section{3.2 The Rank of a Matrix and Matrix Inverses } 
\subsection{The Rank of a Matrix}
%definition of the rank of a matrix
%Note 

%Thm3.3 

%Thm3.4 

%Cor(!!!) (ERO & ECO on matrix...) with one example and exercise 8

%Thm3.5 

%Thm3.6 

%Cor3.6.1 

%Cor3.6.2 

%Cor3.6.3(!!!) (Every invertible matrix is a ...) with one example and exercise 5(c) 

%Thm3.7 with Example 4(a) on Textbook 

\subsection{Matrix Inverses}
%definition of augmented matrix(A|B)

% Formulation with one example and exercise 7

% Example 7 on Textbook 


\section{3.3 Systems of Linear Equations—Theoretical Aspects }
%definiton of several noun on Textbook 

%Take 3 examples to explain their shape of solutions 

%definition of homogeneous and nonhmogeneous 
%Note that homo system at least one solution 

%Thm3.8 with one example 

%Thm3.9 with the example continued from example above(the example of Thm3.8)

%Thm3.10 with one example 

%definition of the augmented matrix of the system Ax=b 

%Thm3.11 with one example 


\section{3.4 Systems of Linear Equations—Computational Aspects } 
%definition of equivalent 

%Thm3.13 

%Cor 

%Example on P.183 of matrix, discribe the matrix solution step by arrows 

%definition of reduce row echlon form with 4 examples(one is right and the other are wrong) 

%definition for Gaussian elimination 

%Thm3.14 with one example on P.187 

%Thm3.15  

\subsection{An interpretation of the Reduce Row Echlon Form}
%Thm3.16 is the results of interpretation of the Reduce Row Echlon Form 

%Cor

%Example 2 on Textbook (了解形狀)

%Example 3 on Textbook (尋找子集S, S是V的基底)

%Example 4 on Textbook(擴張S, 使S變成V的基底) 

%Example (Another solution way for Sec1.6 Example 6 )
%上述四題例題請找「柏亦」拿筆記!

\end{document}
