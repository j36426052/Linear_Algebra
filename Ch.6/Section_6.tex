%% LyX 2.3.5.2 created this file.  For more info, see http://www.lyx.org/.
%% Do not edit unless you really know what you are doing.
\documentclass[12pt,english]{article}
\usepackage{fontspec}
	\setmainfont{Cochin}
\usepackage{setspace}
	\onehalfspace
\usepackage{amssymb, amsmath, amsthm}
\usepackage{mathrsfs}
\usepackage{tasks}
\renewcommand{\ttdefault}{cmtl}
\usepackage{tcolorbox}
\usepackage[framemethod=tikz]{mdframed} % Allows defining custom boxed/framed environments
\usepackage[a4paper]{geometry}
\geometry{verbose,tmargin=2.5cm,bmargin=2.5cm,lmargin=3cm,rmargin=3cm}
\usepackage{enumitem}
\everymath{\displaystyle}

\makeatletter
%%%%%%%%%%%%%%%%%%%%%%%%%%%%%% Textclass specific LaTeX commands.
\newlength{\lyxlabelwidth}      % auxiliary length 
\theoremstyle{definition}
\newtheorem*{defn*}{\protect\definitionname}
\theoremstyle{plain}
\newtheorem*{thm*}{\protect\theoremname}
\newtheorem*{sol*}{solution:}
\newtheorem*{defn}{\underline{Definition}}
%%%%%%%%%%%%%%%%%%%%%%%%%%%%%% User specified LaTeX commands.
\renewcommand{\P}{\mathscr{P}}
\newcommand{\B}{\mathscr{B}}
\newcommand{\A}{\mathscr{A}}
\newcommand{\C}{\mathbb{C}}
\newcommand{\CC}{\mathscr{C}}
\newcommand{\R}{\mathbb{R}}
\newcommand{\Q}{\mathbb{Q}}
\newcommand{\Z}{\mathbb{Z}}
\newcommand{\N}{\mathbb{N}}
\newcommand{\X}{\mathcal{X}}
\newcommand{\T}{\mathscr{T}}
\newcommand{\arbuni}{\bigcup_{\alpha\in I}}
\newcommand{\finint}{\bigcap_{i=1}^n}
\newcommand{\Ua}{{\textsc{U}_\alpha}}
\newcommand{\Ui}{\textsc{U}_i}

\newcommand{\pair}[2]{\left( \,#1\,,\,#2\,\right) }
\makeatother

\usepackage{babel}
\providecommand{\definitionname}{Definition}
\providecommand{\theoremname}{Theorem}
%----------------------------------------------------------------------------------------
%	INFORMATION ENVIRONMENT
%----------------------------------------------------------------------------------------

% Usage:
% \begin{info}[optional title, defaults to "Info:"]
% 	contents
% 	\end{info}

\mdfdefinestyle{info}{%
	topline=false, bottomline=false,
	leftline=false, rightline=false,
	nobreak,
	singleextra={%
		\fill[black](P-|O)circle[radius=0.4em];
		\node at(P-|O){\color{white}\scriptsize\bf i};
		\draw[very thick](P-|O)++(0,-0.8em)--(O);%--(O-|P);
	}
}

% Define a custom environment for information
\newenvironment{info}[1][Info:]{ % Set the default title to "Info:"
	\medskip
	\begin{mdframed}[style=info]
		\noindent{\textbf{#1}}
}{
	\end{mdframed}
}
%----------------------------------------------------------------------------------------
%	WARNING TEXT ENVIRONMENT
%----------------------------------------------------------------------------------------

% Usage:
% \begin{warn}[optional title, defaults to "Warning:"]
%	Contents
% \end{warn}

\mdfdefinestyle{warning}{
	topline=false, bottomline=false,
	leftline=false, rightline=false,
	nobreak,
	singleextra={%
		\draw(P-|O)++(-0.5em,0)node(tmp1){};
		\draw(P-|O)++(0.5em,0)node(tmp2){};
		\fill[black,rotate around={45:(P-|O)}](tmp1)rectangle(tmp2);
		\node at(P-|O){\color{white}\scriptsize\bf !};
		\draw[very thick](P-|O)++(0,-1em)--(O);%--(O-|P);
	}
}

% Define a custom environment for warning text
\newenvironment{warn}[1][Warning:]{ % Set the default warning to "Warning:"
	\medskip
	\begin{mdframed}[style=warning]
		\noindent{\textbf{#1}}
}{
	\end{mdframed}
}
%----------------------------------------------------------------------------------------
%	NUMBERED QUESTIONS ENVIRONMENT
%----------------------------------------------------------------------------------------

% Usage:
% \begin{question}[optional title]
%	Question contents
% \end{question}

\mdfdefinestyle{Example}{
	innertopmargin=1.2\baselineskip,
	innerbottommargin=0.8\baselineskip,
	roundcorner=5pt,
	nobreak,
	singleextra={%
		\draw(P-|O)node[xshift=1em,anchor=west,fill=white,draw,rounded corners=5pt]{%
		Example \exampleTitle};
	}
}

%\newcounter{Question} % Stores the current question number that gets iterated with each new question

% Define a custom environment for numbered questions
\newenvironment{example}[1][\unskip]{
	\bigskip
	%\stepcounter{Question}
	\newcommand{\exampleTitle}{~#1}
	\begin{mdframed}[style=Example]
}{
	\end{mdframed}
	\medskip
}
\parindent=0pt 
\linespread{1.5}
\begin{document}
	\section*{Chap.6 Inner Product Space}		\subsection*{\S 6.1 Inner product and norms}
	
	%Definition for inner product
	
	
	%definition for adjoint 
	
	\begin{example}
		Let $\mathrm{A} , \mathrm{B}\in M_{m\times n}(F)$ ,  $$\langle\mathrm{A}, \mathrm{B}\rangle = \mathrm{Tr}\left(\mathrm{B}^*\mathrm{A}\right)$$ Determine the $\langle\cdot \,,\, \cdot\rangle$ is an inner prduct.
		\begin{sol*}
			$ $
			\begin{itemize}
				\item Claim 1. $\langle\mathrm{A} , \mathrm{A}\rangle \geq 0 $
				\item Claim 2. $\langle\mathrm{A},\mathrm{B}\rangle = \overline{\langle\mathrm{B} , \mathrm{A}\rangle}$
				\item Claim 3. $\langle k\mathrm{A}+\mathrm{B} , \mathrm{C}\rangle = k\langle\mathrm{A} , \mathrm{C}\rangle + \langle\mathrm{B},\mathrm{C}\rangle$
			\end{itemize}
		\end{sol*}
	\end{example}
	\begin{warn}[Notice :]
		A vector space $\mathrm{V}$ over $F$ endowed with a specific inner product is called a inner product space.
		If $F = \C$ we called it "Complex inner product space", whereas if $F = \R$ we called it "real inner product space".  
	\end{warn}
	%Thm 6.1
	%definition for norms on textbook P.339
	%Thm 6.2
	%definition for orthogonal and orthonormal 
	%Exercise 14 17 21 23
	
	\subsection*{\S 6.2 Gram-Schmidt Orthogonalization Process}
	
	\begin{defn}
		$V$ be a inner product space. A subset of $V$ is a orthonormal basis for $V$ if it is an ordered basis that is orthonormal. 
	\end{defn}
	\begin{thm*}
		$V$ be a inner product space , $S = \{v_1 , \cdots , v_k\}$ be a orthogonal subset of $V$ consisting of nonzero vectors. If $y \in \mathrm{Span}(S)$ then 
		
		\[y = \sum_{i=1}^{k}\frac{\langle y,v_i\rangle}{\|v_i\|^2}\cdot v_i \]	
	\end{thm*}
	\begin{proof} $ $\\
		$\because y ~\in \mathrm{Span}(S)\implies y = \sum_{i = 1}^{k} a_iv_i $ where $a_1 , \cdots , v_k ~\in F$
		
		For $1\leq i \leq k$ , $\langle y , v_i \rangle = \langle \sum_{i=1}^k a_iv_i , v_j \rangle = \sum_{i=1}^{K}a_i\langle v_i , v_j \rangle\ldots(1)$
		
		$\because \beta $ is an orthogonal subset $\implies \langle v_i , v_j \rangle = 0$ , $\forall i , j$ , $i\neq j$ $\therefore$ (1) $ = a_j\langle v_j , v_j \rangle = a_j\|v_j\|^2 $
		
		$\implies $ for $1\leq j \leq k , \langle y , v_j \rangle = a_j \|v_j\|^2 \implies a_j = \frac{\langle y , v_j \rangle}{\|v_j\|^2}$
		
		$y = \sum_{j=1}^{k} a_jv_j = y = \sum_{j=1}^{k} \frac{\langle y , v_j \rangle}{\|v_j\|^2} \cdot v_i $
		
	\end{proof}
		
	$\bullet$ Corollary : 
		\begin{enumerate}
			\item If $S$ is orthonormal and $y \in \mathrm{Span}(S)$ , $ y= \sum_{i=1}^{k} \langle y , v_i \rangle \cdot v_i$
			\item $V$ : inner product space , $S$ : orthogonal subset of $V$ consisting nonzero vectors , then $S$ is linearly independent.
		\end{enumerate}
		
		The main idea is that $\{w_1 , w_2\}$ is a subset of inner product space we want to construct an orthogonal subset from linear independent sets of vectors is such the way both set generate the same space.
		
		%Thm 6.4
		
		%Thm 6.5
		
		\begin{example}
			$V = \R^3$ , $S = \left\{(1,0,1) , (0 , 1, 1) , (1,3,3)\right\}$ and $x = (1,1,2)$  
		\end{example}	
	
\end{document}