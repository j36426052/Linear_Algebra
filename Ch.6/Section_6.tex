\input{settings.tex}

\begin{document}
	\section*{Chap.6 Inner Product Space}		\subsection*{\S 6.1 Inner product and norms}
	
	%Definition for inner product
	
	
	%definition for adjoint 
	
	\begin{example}
		Let $\mathrm{A} , \mathrm{B}\in M_{m\times n}(F)$ ,  $$\langle\mathrm{A}, \mathrm{B}\rangle = \mathrm{Tr}\left(\mathrm{B}^*\mathrm{A}\right)$$ Determine the $\langle\cdot \,,\, \cdot\rangle$ is an inner prduct.
		\begin{sol*}
			$ $
			\begin{itemize}
				\item Claim 1. $\langle\mathrm{A} , \mathrm{A}\rangle \geq 0 $
				\item Claim 2. $\langle\mathrm{A},\mathrm{B}\rangle = \overline{\langle\mathrm{B} , \mathrm{A}\rangle}$
				\item Claim 3. $\langle k\mathrm{A}+\mathrm{B} , \mathrm{C}\rangle = k\langle\mathrm{A} , \mathrm{C}\rangle + \langle\mathrm{B},\mathrm{C}\rangle$
			\end{itemize}
		\end{sol*}
	\end{example}
	\begin{warn}[Notice :]
		A vector space $\mathrm{V}$ over $F$ endowed with a specific inner product is called a inner product space.
		If $F = \C$ we called it "Complex inner product space", whereas if $F = \R$ we called it "real inner product space".  
	\end{warn}
	%Thm 6.1
	%definition for norms on textbook P.339
	%Thm 6.2
	%definition for orthogonal and orthonormal 
	%Exercise 14 17 21 23
	
	\subsection*{\S 6.2 Gram-Schmidt Orthogonalization Process}
\end{document}