\input{settings.tex}
\begin{document}
	\section*{Chapter 2 Linear Transformation}	
	\subsection*{\S\,2-1 Linear transformation}
\begin{defn}[Definition of Linear Transformation]
$ $\\	Let $\mathrm{V}$ and $\mathrm{W}$ be vector spaces (over $\mathrm{F}$). We call a function $\mathrm{T}: \mathrm{V}  \rightarrow  \mathrm{W}$    a linear transformation from $\mathrm{V}$ to $\mathrm{W}$ if, for all x, y $\in$ $\mathrm{V}$ and c $\in$ $\mathrm{F}$, we have

\begin{enumerate}
	\item [(a)]   $\mathrm{T}$(x + y) = $\mathrm{T}$(x) + $\mathrm{T}$(y) 
	\item [(b)]    $\mathrm{T}$($\mathrm{c}$x) = $\mathrm{c}\mathrm{T}$(x)
\end{enumerate} 
\end{defn}

\begin{defn}[Definition of nullity and rank]
$ $\\	Let $\mathrm{V}$ and $\mathrm{W}$ be vector spaces and let $\mathrm{T}: \mathrm{V}  \rightarrow  \mathrm{W}$ be linear. If N(T) and R(T) are finite-dimensional, then we define the nullity of T, denoted nullity(T), and the rank of T, denoted rank(T), to be the dimensions of N(T) and R(T), respectively.
\end{defn}

\begin{thm*}
$ $ \\ 	Let V and W be vector spaces and $\mathrm{T}: \mathrm{V}  \rightarrow  \mathrm{W}$ be linear. Then N(T) and R(T) are subspaces of $\mathrm{V}$ and $\mathrm{W}$, respectively.
\end{thm*}

\begin{thm*}
$ $\\
 Let V and W be vector spaces, and let $\mathrm{T} : \mathrm{V}  \rightarrow \mathrm{W}$ be linear. If $\beta = \{ v_1,v_2,...,v_n \}$ is a basis for $\mathrm{V}$ , then $ \mathrm{R}(\mathrm{T}) = \mathrm{span}(\mathrm{T}(\beta)) = \mathrm{span}(\{\mathrm{T}(v_1), \mathrm{T}(v_2), . . . , \mathrm{T}(v_n)\})$.

\end{thm*}

\begin{example}[11] %Example 11
	Question : \\
	Let T : $\mathrm{P}_2(R) \rightarrow  \mathrm{P}_3(R) $ be the linear trans formation defined by 
	
	
	 \[T(f(x)) = 2f'(x) + \int_0^x3 f(t) dt \]
	
		\begin{sol*} 
$ $ \\ 	
Now $\mathrm{R}(\mathrm{T}) = \mathrm{Span}( \{ \mathrm{T}(1), \mathrm{T}(x), \mathrm{T}(x^2 )  ) = \mathrm{Span}( \{ 3x,  2+ \frac{3}{2}  x^2 , 4x + x^3 \} ) $. 

 \noindent Since \{ $3x$ , $ 2 + \frac{3}{2} x^2$, $4x + x^3$ \} is linearly independent, rank($\mathrm{T}$) = 3. Since
2
$\dim(P_3(\R)) = 4$ , T is not onto. From the dimension theorem (Thm 2.3) , nullity(T) +
3 = 3. So nullity(T) = 0, and therefore, N(T) = {0}. We conclude from Theorem 2.4 that T is one-to-one.

	\end{sol*}

	\end{example}
\begin{example}[12] %Example 12
	 Question : \\ Let $\mathrm{T}: \mathrm{F}^2 \rightarrow \mathrm{F}^2 $ be the linear transformation defined by 
	 
	 \[T(a_1, a_2) = (a_1 + a_2, a_1)\]

	\begin{sol*} 
$ $ \\
		It is easy to see that N(T) = \{ 0 \} ; so T is one-to-one. Hence Theorem 2.5 tells us that T must be onto. 
	\end{sol*}

\end{example}

\begin{thm*}[2.3 Dimension Theorem]
$ $ \\	
Let V and W be vector spaces,
and let $ \mathrm{T} : \mathrm{V}  \rightarrow  \mathrm{W} $ be linear. \\ If V is finite-dimensional, then
 
 
 \[\rm	nullity(T) + rank(T) = dim(V).\]


\end{thm*}

\begin{thm*}[2.4]
$ $ \\	Let V and W be vector spaces, and let $\rm T: V  \rightarrow  W$ be
linear. Then T is one-to-one if and only if N(T) = {0 }.

\end{thm*}

\begin{thm*}[2.5]
$ $ \\	 Let V and W be vector spaces of equal (finite) dimension, and let $\rm T: V  \rightarrow  W$ be linear. Then the following are equivalent.
\begin{enumerate} 
	\item[(a)] T is one-to-one. 
	\item[(b)] T is onto.
	\item[(c)] rank(T) = dim(V).

\end{enumerate}

\end{thm*}

\begin{example}
	Let $\rm T : \R^2\longrightarrow \R^3 $ , $\rm T(a_1,a_2) = (a_1+a_2 , 0 , 2a_1-a_2)$ , determine that $\rm T $ is linear , 1-1 , onto or not.
	\begin{sol*}
		 $ $
		\begin{itemize}
			\item Calim : $\rm T$ is linear.\newline
			Let $ x =(a_1,a_2) \,,\, y= (b_1,b_2)$ 
				
				\(\begin{aligned} \rm T(c x+y) &=\rm T\left(c\left(a_{1}, a_{2}\right)+\left(b_{1}, b_{2}\right)\right)= T\left(ca_{1}+b_{1}, ca_{2}+b_{2}\right) \\ &=\rm \left(c a_{1}+b_{1}+c a_{2}+b_{2}, 0,2 c a_{1}+2 b_{1}-ca_{2}-b_{2}\right) \\ &=\rm \left(c\left(a_{1}+a_{2}\right)+\left(b_{1}+b_{2}\right), 0, c\left(2 a_{1}-a_{2}\right)+\left(2 b_{1}-b_{2}\right)\right) \\ &=\rm c\left(a_{1}+a_{2}, 0,2 a_{1}-a_{2}\right)+\left(b_{1}+b_{2}, 0,2 b_{1}-b_{2}\right) \\ &=\rm c T(x)+T (y) \end{aligned}\)
				
			$\therefore \rm T $ is linear
			\item 1-1 and onto 	
			
			By Thm 2.2 in text book choose a basis for $\R^2 , \beta = \{(1,0) , (0,1)\}$
			
			$\rm R(T) = span\left(T(\beta)\right) = span(\left\{T(1,0) , T(0,1)\right\}) = span\left((1,0,-1), (1,0,-2)\right)$ 
			
			Clearly it is L.I. $\rm\implies rank(T) = 2$ , and apply the Dimension Theorem $\rm rank(T)=2\neq3=\dim(\R^3) $, and $\rm nullitily(T) = 1 $ , then it is not onto.
			
			By Thm 2.4  it is not one to one.
			
		\end{itemize}
	\end{sol*}
		\end{example}
		\begin{example}
		Let \(\rm V\) and \(\rm W\) be vector spaces, let \(\mathrm{T}:\rm  V \rightarrow W\) be linear, and let
\(\left\{w_{1}, w_{2}, \ldots, w_{k}\right\}\) be a linearly independent subset of \(\rm R(T)\). Prove that
if \(S=\left\{v_{1}, v_{2}, \ldots, v_{k}\right\}\) is chosen so that \(\mathrm{T}\left(v_{i}\right)=w_{i}\) for \(i=1,2, \ldots, k\),
then \(S\) is linearly independent.
\begin{sol*}$ $
Calim : $\sum_{i=1}^n a_iv_i = 0\implies a_1 = a_2 = a_3 =\cdots =a_n = 0 $
	
	Let \(\sum_{i=1}^{n} a_{i} v_{i}=0\)
then \(\mathrm{T}\left(\sum_{i=1}^{n} a_{i} v_{i}\right)=0\)
Since \(\rm T\) is linear , \(\mathrm{T}\left(\sum_{i=1}^{n} a_{i} v_{i}\right)=\sum_{i=1}^{n} a_{i} \mathrm{T}\left(v_{i}  \right)=\sum_{i=1}^na_iw_i = 0\)
Since $\mathrm{S}$ is L.I. $\implies a_1 = a_2=\cdots = a_n = 0$ 
\end{sol*}
	

\end{example}
\begin{thm*}[2.6]
$ $ \\Let V and W be vector spaces over F, and suppose that \{ $v_1,v_2,...,v_n$ \} is a basis for V. For $w_1,w_2,...,w_n$ in W, there exists exactly one linear transformation $\mathrm{T}: \mathrm{V}  \rightarrow \mathrm{W}$ such that T($v_i$) = $w_i$ for i = 1,2,...,n.
\end{thm*}
		
		
\end{document}